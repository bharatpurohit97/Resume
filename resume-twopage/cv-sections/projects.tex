% ----------------------------------------------------------------------------------------
% SECTION TITLE
% ----------------------------------------------------------------------------------------
\cvsection{Projects}

% ----------------------------------------------------------------------------------------
% SECTION CONTENT
% ----------------------------------------------------------------------------------------

\begin{cventries}

  % ------------------------------------------------
  \cventry
  {Research Project, Prof. Sumita Kachhwaha}
  {\href{https://github.com/DBT-BIF/Biosensor-Python}{\entrytitlestyle{Biosensor-Python}}
    {}}
  {DBT-BIF Jaipur}
  {July. 2018 - Sept. 2018}
  {
    \begin{cvitems}
    \item Created a web enabled tool for finding unique DNA fragment that also draw phylogeny
     of life.
    \item Designed a module in python for Extraction, Parsing, Pre-processing, Sequencing, Searching of Genomic Database and Networking for phylogeny of life.
    \item Used Machine learning and statistical inference for classification and prediction in Genomic database
    \end{cvitems}
  }
  \cventry
  {Course Project, Prof. Abhilasha Dangi}
  {\entrytitlestyle{Senior Students Placement Cell}}
  {University of Rajasthan, Jaipur}
  {Feb. 2017 - May. 2017}
  {
    \begin{cvitems}
      \item Conceptualized new web 2.0 based IT services such as Interest Group Discussion Forums,
       Senior alumni Information and contacts, Campus Wiki, Project Database
      \item Allow easier collaboration, better information organization and formalizing
       undocumented technical.
      \item Designed intuitive user interfaces and work-flows for these services and
       customization then finally deploy these services in the institute
    \end{cvitems}
  }
    
  \cventry
  {Research Project, Team of 2 members}
  {\entrytitlestyle{Thought Reading Device: Predicting Text by EEG signals (BCI)}}
  {IIT Guwahati}
  {Nov. 2016 - March. 2017}
  {
    \begin{cvitems}
      \item Designed Computational model to infer psychological state of human brain with one colleague.
      \item used BrainIAK software that allows for decoding digital brain data to reveal
      how neural activity give rises to learning, memory and other cognitive function.
      \item Used viterbi force alignment and hidden markov model for mapping phone sequences to word. 
    \end{cvitems}
  }

%  \cventry
%  {Individual Project, Dr. Rakesh Sharma}
%  {\entrytitlestyle{PageMatrix : Co-founder}}
%  {Jaipur}
%  {May. 2018 - Aug. 2018}
%  {
%    \begin{cvitems}
%      \item Designed and Developed a SEO Tool for Page-ranking, Links and Emails 
%        Extraction and Keyword prediction for websites with team of three members.
%      \item Build Python scripts and used Django, Shell scripting to set up Frontend, Backend
%        and Sqlite3 for database.
%      \item Deployed on web for 3 months, with 1800+ users and 45 matches.
%    \end{cvitems}
%  }

  \cventry
  {Leader, Team SudoHack Project}
  {\href{https://github.com/LNMHacks/LNMHacks-3.0-Submission/sudo hack}{\entrytitlestyle{Pneumonia Detection: Deep learning}}}
  {CCT Jaipur}
  {Sep. 2018 - Nov. 2018}
  {
    \begin{cvitems}
      \item Build an algorithm that shows good confidence interval in detecting pneumonia
        and Automatically locate lung opacities on chest radiographs and medical images. 
      \item Used TensorFow, CNN model: DenseNet169 for transfer learning,
        and predict lung opacity by loading models.
      \item Finished in top 10 teams from all over India at LNMHacks 3.0.
    \end{cvitems}
  }
 


  % ------------------------------------------------
\end{cventries}

%%% Local Variables:
%%% mode: xelatex
%%% TeX-master: "../resume_twopage.tex"
%%% End:
